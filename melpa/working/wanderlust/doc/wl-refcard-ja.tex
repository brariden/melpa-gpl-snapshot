% Reference Card for Wanderlust
% The format of this file is adapted from the GNU Emacs reference card
%
% You can compile with:
% ptex wl-refcard-ja.tex
% dvips wl-refcard-ja.dvi -t landscape -o wl-refcard-ja.ps
% pstops '2:0U(21cm,29.7cm),1' wl-refcard-ja.ps wl-refcard-ja-b.ps

% Reference Card for GNU Emacs version 20 on Unix systems
%**start of header
\newcount\columnsperpage

% This file can be printed with 1, 2, or 3 columns per page (see below).
% Specify how many you want here.  Nothing else needs to be changed.

\columnsperpage=3

% Copyright (c) 1987, 1993, 1996, 1997 Free Software Foundation, Inc.

% This file is part of GNU Emacs.

% GNU Emacs is free software; you can redistribute it and/or modify
% it under the terms of the GNU General Public License as published by
% the Free Software Foundation; either version 2, or (at your option)
% any later version.

% GNU Emacs is distributed in the hope that it will be useful,
% but WITHOUT ANY WARRANTY; without even the implied warranty of
% MERCHANTABILITY or FITNESS FOR A PARTICULAR PURPOSE.  See the
% GNU General Public License for more details.

% You should have received a copy of the GNU General Public License
% along with GNU Emacs; see the file COPYING.  If not, write to
% the Free Software Foundation, Inc., 59 Temple Place - Suite 330,
% Boston, MA 02111-1307, USA.

% This file is intended to be processed by plain TeX (TeX82).
%
% The final reference card has six columns, three on each side.
% This file can be used to produce it in any of three ways:
% 1 column per page
%    produces six separate pages, each of which needs to be reduced to 80%.
%    This gives the best resolution.
% 2 columns per page
%    produces three already-reduced pages.
%    You will still need to cut and paste.
% 3 columns per page
%    produces two pages which must be printed sideways to make a
%    ready-to-use 8.5 x 11 inch reference card.
%    For this you need a dvi device driver that can print sideways.
% Which mode to use is controlled by setting \columnsperpage above.
%
% Author:
%  Stephen Gildea
%  Internet: gildea@stop.mail-abuse.org
%
% Thanks to Paul Rubin, Bob Chassell, Len Tower, and Richard Mlynarik
% for their many good ideas.

% If there were room, it would be nice to see a section on Dired.

\input version.tex
\def\year{2001}

\def\shortcopyrightnotice{\vskip 1ex plus 2 fill
  \centerline{\small \copyright\ \year\ Yuuichi Teranishi
  Permissions on back.}}

\def\copyrightnotice{
\vskip 1ex plus 2 fill\begingroup\small
\centerline{Copyright \copyright\ \year\  Yuuichi Teranishi}
\centerline{for Wanderlust \versionnumber}

Permission is granted to make and distribute copies of
this card provided the copyright notice and this permission notice
are preserved on all copies.

\endgroup}

% make \bye not \outer so that the \def\bye in the \else clause below
% can be scanned without complaint.
\def\bye{\par\vfill\supereject\end}

\newdimen\intercolumnskip	%horizontal space between columns
\newbox\columna			%boxes to hold columns already built
\newbox\columnb

\def\ncolumns{\the\columnsperpage}

\message{[\ncolumns\space 
  column\if 1\ncolumns\else s\fi\space per page]}

\def\scaledmag#1{ scaled \magstep #1}

% This multi-way format was designed by Stephen Gildea October 1986.
% Note that the 1-column format is fontfamily-independent.
\if 1\ncolumns			%one-column format uses normal size
  \hsize 4in
  \vsize 10in
  \voffset -.7in
  \font\titlefont=\fontname\tenbf \scaledmag3
  \font\headingfont=\fontname\tenbf \scaledmag2
  \font\smallfont=\fontname\sevenrm
  \font\smallsy=\fontname\sevensy

  \footline{\hss\folio}
  \def\makefootline{\baselineskip10pt\hsize6.5in\line{\the\footline}}
\else				%2 or 3 columns uses prereduced size
  \hsize 3.2in
  \vsize 7.95in
  \hoffset -.75in
  \voffset -.745in
  \font\titlefont=cmbx10 \scaledmag2
  \font\headingfont=cmbx10 \scaledmag1
  \font\smallfont=cmr6
  \font\smallsy=cmsy6
  \font\eightrm=cmr8
  \font\eightbf=cmbx8
  \font\eightit=cmti8
  \font\eighttt=cmtt8
  \font\eightmi=cmmi8
  \font\eightsy=cmsy8
  \textfont0=\eightrm
  \textfont1=\eightmi
  \textfont2=\eightsy
  \def\rm{\eightrm}
  \def\bf{\eightbf}
  \def\it{\eightit}
  \def\tt{\eighttt}
  \normalbaselineskip=.8\normalbaselineskip
  \normallineskip=.8\normallineskip
  \normallineskiplimit=.8\normallineskiplimit
  \normalbaselines\rm		%make definitions take effect

  \if 2\ncolumns
    \let\maxcolumn=b
    \footline{\hss\rm\folio\hss}
    \def\makefootline{\vskip 2in \hsize=6.86in\line{\the\footline}}
  \else \if 3\ncolumns
    \let\maxcolumn=c
    \nopagenumbers
  \else
    \errhelp{You must set \columnsperpage equal to 1, 2, or 3.}
    \errmessage{Illegal number of columns per page}
  \fi\fi

  \intercolumnskip=.46in
  \def\abc{a}
  \output={%			%see The TeXbook page 257
      % This next line is useful when designing the layout.
      %\immediate\write16{Column \folio\abc\space starts with \firstmark}
      \if \maxcolumn\abc \multicolumnformat \global\def\abc{a}
      \else\if a\abc
	\global\setbox\columna\columnbox \global\def\abc{b}
        %% in case we never use \columnb (two-column mode)
        \global\setbox\columnb\hbox to -\intercolumnskip{}
      \else
	\global\setbox\columnb\columnbox \global\def\abc{c}\fi\fi}
  \def\multicolumnformat{\shipout\vbox{\makeheadline
      \hbox{\box\columna\hskip\intercolumnskip
        \box\columnb\hskip\intercolumnskip\columnbox}
      \makefootline}\advancepageno}
  \def\columnbox{\leftline{\pagebody}}

  \def\bye{\par\vfill\supereject
    \if a\abc \else\null\vfill\eject\fi
    \if a\abc \else\null\vfill\eject\fi
    \end}  
\fi

% we won't be using math mode much, so redefine some of the characters
% we might want to talk about
\catcode`\^=12
\catcode`\_=12

\chardef\\=`\\
\chardef\{=`\{
\chardef\}=`\}

\hyphenation{mini-buf-fer}

\parindent 0pt
\parskip 1ex plus .5ex minus .5ex

\def\small{\smallfont\textfont2=\smallsy\baselineskip=.8\baselineskip}

% newcolumn - force a new column.  Use sparingly, probably only for
% the first column of a page, which should have a title anyway.
\outer\def\newcolumn{\vfill\eject}

% title - page title.  Argument is title text.
\outer\def\title#1{{\titlefont\centerline{#1}}\vskip 1ex plus .5ex}

% section - new major section.  Argument is section name.
\outer\def\section#1{\par\filbreak
  \vskip 3ex plus 2ex minus 2ex {\headingfont #1}\mark{#1}%
  \vskip 2ex plus 1ex minus 1.5ex}

\newdimen\keyindent

% beginindentedkeys...endindentedkeys - key definitions will be
% indented, but running text, typically used as headings to group
% definitions, will not.
\def\beginindentedkeys{\keyindent=1em}
\def\endindentedkeys{\keyindent=0em}
\endindentedkeys

% paralign - begin paragraph containing an alignment.
% If an \halign is entered while in vertical mode, a parskip is never
% inserted.  Using \paralign instead of \halign solves this problem.
\def\paralign{\vskip\parskip\halign}

% \<...> - surrounds a variable name in a code example
\def\<#1>{{\it #1\/}}

% kbd - argument is characters typed literally.  Like the Texinfo command.
\def\kbd#1{{\tt#1}\null}	%\null so not an abbrev even if period follows

% beginexample...endexample - surrounds literal text, such a code example.
% typeset in a typewriter font with line breaks preserved
\def\beginexample{\par\leavevmode\begingroup
  \obeylines\obeyspaces\parskip0pt\tt}
{\obeyspaces\global\let =\ }
\def\endexample{\endgroup}

% key - definition of a key.
% \key{description of key}{key-name}
% prints the description left-justified, and the key-name in a \kbd
% form near the right margin.
\def\key#1#2{\leavevmode\hbox to \hsize{\vtop
  {\hsize=.75\hsize\rightskip=1em
  \hskip\keyindent\relax#1}\kbd{#2}\hfil}}

\newbox\metaxbox
\setbox\metaxbox\hbox{\kbd{M-x }}
\newdimen\metaxwidth
\metaxwidth=\wd\metaxbox

% metax - definition of a M-x command.
% \metax{description of command}{M-x command-name}
% Tries to justify the beginning of the command name at the same place
% as \key starts the key name.  (The "M-x " sticks out to the left.)
\def\metax#1#2{\leavevmode\hbox to \hsize{\hbox to .75\hsize
  {\hskip\keyindent\relax#1\hfil}%
  \hskip -\metaxwidth minus 1fil
  \kbd{#2}\hfil}}

% threecol - like "key" but with two key names.
% for example, one for doing the action backward, and one for forward.
\def\threecol#1#2#3{\hskip\keyindent\relax#1\hfil&\kbd{#2}\hfil\quad
  &\kbd{#3}\hfil\quad\cr}

%**end of header


\title{Wanderlust Reference Card}

\centerline{(for version \versionnumber)}

\section{Wanderlust $B$N5/F0(B}

\key{Wanderlust $B$r5/F0$9$k(B}{M-x wl}
\key{$B%a%C%;!<%8$r:n@.(B}{M-x wl-draft}

\section{Folder Mode}

\key{Wanderlust $B$r$d$a$k(B}{q}
\key{$B<!(B}{n}
\key{$BA0(B}{p}
\key{$B<!$NL$FI(B}{N}
\key{$BA0$NL$FI(B}{P}
\key{$B;XDj$7$?%U%)%k%@$KF~$k(B}{g}
\key{$B$3$N%U%)%k%@$KF~$k(B}{SPC}
\key{$B%0%k!<%W$N3+JD(B}{SPC}
\key{$B%"%/%;%9%0%k!<%W$N:F%9%-%c%s(B}{C-u SPC}
\key{$B?75,%a%C%;!<%8$r%A%'%C%/(B}{s}
\key{$BF14|(B}{S}
\key{$B%W%j%U%'%C%A(B}{I}
\key{$BA4$F4{FI$K(B}{c}
\key{$B%(%/%9%Q%$%"(B}{e}
\key{$B%a%C%;!<%8$N:n@.(B}{w}
\key{$B%a%C%;!<%8$N:n@.(B($B08@h$r?dB,(B)}{W}
\key{$B%"%I%l%9D"$N:FFI9~(B}{Z}
\key{$B$4$_H"$r6u$K(B}{E}

\section{folder management}

\key{$B%+%C%H(B}{C-k}
\key{$BNN0h$r%+%C%H(B}{C-w}
\key{$BE=$jIU$1(B}{C-y}
\key{$B%U%)%k%@$r2C$($k(B}{m a}
\key{$B%0%k!<%W$r2C$($k(B}{m g}
\key{$B%"%/%;%9%0%k!<%W$r2C$($k(B}{m A}

\section{access group}

\key{$BHs9VFI%U%)%k%@$bI=<((B}{L}
\key{$BHs9VFI%U%)%k%@$r1#$9(B}{l}
\key{$B9VFI$r$d$a$k(B}{u}
\key{$B9VFI$r$d$a$k(B($BNN0h(B)}{U}

\newcolumn
\section{Summary Mode}

\key{$B%U%)%k%@%P%C%U%!$KLa$k(B}{q}
\key{$B;XDj$7$?%U%)%k%@$KF~$k(B}{g}
\key{$B<!(B}{n}
\key{$BA0(B}{p}
\key{$B<!$NL$FI(B}{N}
\key{$BA0$NL$FI(B}{P}
\key{$B@hF,$N%a%C%;!<%8(B}{<}
\key{$BKvHx$N%a%C%;!<%8(B}{>}
\key{$BD>A0$K8+$?%a%C%;!<%8(B}{TAB}
\key{$B%a%C%;!<%8$rFI$`(B}{SPC}
\key{$B%a%C%;!<%8$N:FFI9~(B}{C-u .}
\key{$B%a%C%;!<%8I=<($N%H%0%k(B}{v}
\key{$B%a%C%;!<%8$r:n@.(B}{w}
\key{$B%a%C%;!<%8$r:n@.(B($B08@h$r?dB,(B)}{W}
\key{$BJV?.(B}{a}
\key{$B0zMQ$7$FJV?.(B}{A}
\key{$B%U%)%o!<%I(B}{f}
\key{$B:FJT=8(B}{E}
\key{$B0u:~(B}{\#}
\key{$B%a%C%;!<%8%P%C%U%!$X(B}{j}
\key{$B%"%I%l%9D"$N:FFI9~(B}{Z}

\section{mark command}

\key{$B=hJ,%^!<%/$rIU$1$k(B}{d}
\key{$B:o=|%^!<%/$rIU$1$k(B}{D}
\key{$B%j%U%!%$%k%^!<%/$rIU$1$k(B}{o}
\key{$B%3%T!<%^!<%/$rIU$1$k(B}{O}
\key{$B%W%j%U%'%C%A%^!<%/$rIU$1$k(B}{i}
\key{$B:FAw%^!<%/$rIU$1$k(B}{\~{}}
\key{$B=EMW%^!<%/$rIU$1$k(B}{\$}
\key{$BI8E*%^!<%/$rIU$1$k(B}{*}
\key{$BA4$F$N%a%C%;!<%8$rI8E*$K$9$k(B}{m a}
\key{$B%9%l%C%I$rI8E*$K$9$k(B}{m t}
\key{$B%a%C%;!<%8$rA*$S=P$7$FI8E*$K(B}{?}
\key{$BI8E*$NCf$+$i%a%C%;!<%8$rA*$S=P$9(B}{m *}
\key{$BI8E*$r%j%U%!%$%k(B}{m o}
\key{$BI8E*$r:o=|(B}{m d}
\key{$BI8E*$r0u:~(B}{m \#}
\key{$BI8E*$N3F!9$r%Q%$%W$KN.$9(B}{m |}
\key{$B0l;~E*%^!<%/$N%"%/%7%g%s$r<B9T(B}{x}

\section{prefix arguments for marking}

\key{$BI8E*$KBP$7$F(B}{m}
\key{$B%+!<%=%k0J9_$N%9%l%C%I$KBP$7$F(B}{t}
\key{$BNN0h$KBP$7$F(B}{r}

\newcolumn
\section{sticky summary}

\key{$B%5%^%j$r%9%F%#%C%-!<$K(B}{M-s}
\key{$B;XDj$7$?%U%)%k%@$KF~$k(B}{G}
\key{$B<!$N%5%^%j(B}{C-c C-n}
\key{$BA0$N%5%^%j(B}{C-c C-p}
\key{$B%5%^%j$rGK4~(B}{C-u q}

\section{Message Mode}

\key{$B%5%^%j%P%C%U%!$KLa$k(B}{q}
\key{$B%9%/%m!<%k$+<!$N%Q!<%H(B}{SPC}
\key{$B%9%/%m!<%k$+A0$N%Q!<%H(B}{DEL}
\key{$B<!$N%Q!<%H(B}{n}
\key{$BA0$N%Q!<%H(B}{p}
\key{$B%Q!<%H$r:F@8(B}{v}
\key{$B%Q!<%H$r<h$j=P$9(B}{e}
\key{$B%Q!<%H$rI=<((B}{C-c C-v C-c}
\key{$B%Q!<%H$r1#$9(B}{C-c C-d C-c}
\key{MIME-View $B$N%X%k%W(B}{?}

\section{Draft Mode}

\key{$B%a%C%;!<%8$r0zMQ(B}{C-c C-y}
\key{$B%F%s%W%l!<%H$NA^F~(B}{C-c C-j}
\key{$B=pL>$NA^F~(B}{C-c C-w}
\key{$BNN0h$r>J$/(B}{C-c C-d}
\key{$B%W%l%S%e!<(B}{C-c C-p}
\key{$BAw?.(B}{C-c C-c}
\key{$BAp9F$rJ]B8(B}{C-c C-z}
\key{$BAp9F$rGK4~(B}{C-c C-k}
\key{$BB>$N%I%i%U%H$X(B}{C-c C-o}
\key{$B%U%!%$%k$rE:IU(B}{C-c C-x TAB}
%\key{Insert a reference to external body}{C-c C-x C-e}
\key{MIME $B%?%0$NA^F~(B}{C-c C-x t}
%\key{Insert a text message}{C-c C-x C-t}
\key{$B%a%C%;!<%8$r0E9f2=(B}{C-c C-x e}
\key{$B%a%C%;!<%8$K=pL>(B}{C-c C-x s}
\key{MIME-Edit $B$N%X%k%W(B}{C-c C-x ?}

\shortcopyrightnotice

\newcolumn
\section{Plugged Status}

\key{$B%W%i%0>uBV$N%H%0%k(B($BA4BN(B)}{M-t}
\key{$B%W%i%0%I%b!<%I$X$NF~B`=P(B}{C-t}
\key{$B%W%i%0>uBV$N%H%0%k(B($B8DJL(B)}{SPC}

\section{Address Book Management}

\key{$B%"%I%l%9%^%M!<%8%c$N5/F0(B}{C-c C-a}
\key{$B%"%I%l%9%^%M!<%8%c$N=*N;(B}{q}
\key{{\tt To:} $B$K;XDj(B}{t}
\key{{\tt Cc:} $B$K;XDj(B}{c}
\key{{\tt Bcc:} $B$K;XDj(B}{b}
\key{$B08@h$N;XDj$r2r=|(B}{u}
\key{$B;XDj$K1h$C$F%a%C%;!<%8$r:n@.(B}{x}
\key{$B?75,EPO?(B}{a}
\key{$BJT=8(B}{e}
\key{$B:o=|(B}{d}

\copyrightnotice

\bye

% Local variables:
% compile-command: "tex refcard"
% End:
