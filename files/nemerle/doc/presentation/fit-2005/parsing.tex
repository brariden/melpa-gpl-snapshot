\documentclass[10pt]{beamer}
\usepackage{beamerthemeshadow}
%\usepackage{beamerthemesidebar}
\usepackage{graphicx}
\usepackage{polski}
\usepackage[utf8]{inputenc}

\newcommand{\net}[0]{{\tt .NET}}
\newcommand{\kw}[1]{{\textcolor{kwcolor}{\tt #1}}}
\newcommand{\ra}{\texttt{ -> }}

\definecolor{kwcolor}{rgb}{0.2,0.4,0.0}
\definecolor{lgray}{rgb}{0.8,0.8,0.8}

\title{Parsowanie dynamicznie rozszerzalnej składni}
\author{Kamil Skalski}
\institute{Forum Informatyki Teoretycznej \\ Karpacz}
\date{16 kwietnia 2005}


\begin{document}

\section{Dalekowzroczny język}

\frame{\titlepage}

\frame{
\frametitle{Rozszerzenia języków programowania}
\begin{itemize}
  \item języki powinny być tworzone z myślą o rozwoju
  \item użytkownik sam rozszerza język
  \item nowa semantyka istniejących konstrukcji - makra
  \item nowa składnia do nowych konstrukcji - rozszerzenia składniowe
\end{itemize}
}

\frame{
Chcemy aby makra były bezpieczne:
\begin{itemize}
  \item higiena generowanych nazw
  \item generacja kodu jest oparta o algebraiczne struktury danych, a nie tekstualne
    operacje
  \item rozszerzenia składniowe muszą być pod kontrolą - dodanie nowej składni
    nie może mieć wpływu na istniejący kod, zatem powinny znajdować się we własnych
    przestrzeniach nazw, które możemy na żądanie otworzyć
\end{itemize}
}

\section{Metaprogramowanie}

\frame[containsverbatim]{
\frametitle{Co to jest makro?}
  \emph{Makro} to funkcja o sygnaturze $f : \epsilon \rightarrow \epsilon$, gdzie 
  $\epsilon$ to drzewo składniowe w reprezentacji używanej przez kompilator.\\
  \vspace*{2mm}
  Makra są wykonywane przez kompilator w czasie analizy drzewa programu,
  umożliwiając zmianę, generację i analizę kodu w łatwy sposób.\\
  \vspace*{2mm}
  Kompilator uruchamia makro po napotkaniu wyrażenia o postaci 
  \begin{verbatim}
    nazwa-makra (parametry)
  \end{verbatim}
  lub składni dodanej przez załadowane
  makra.
}

\frame[containsverbatim]{
\frametitle{Przykład}
Deklaracja makra
\begin{verbatim}
  macro repeat_times (count, body)
  syntax ("repeat", "(", count, ")", body)
  {
    <[ for (mutable i = 0; i < $count; i++) $body ]>
  }
\end{verbatim}

i użycie

\begin{verbatim}
  ...
  repeat (10) {
    print ("Ala")
  }
  ...
\end{verbatim}
}

\section{Pierwsze podejście}

\frame{
\frametitle{Jak parsować rozszerzalną składnię?}
\begin{itemize}
  \item prosty system wymagający słowa kluczowego lub operatora rozpoczynającego
        rozszerzenie składniowe
  \item parser rozumie polecenia otwarcia modułów i wprowadza nowe słowa
        kluczowe do analizy leksykalnej
  \item dynamicznie budujemy drzewo rozszerzeń - po załadowaniu składni z nowej
        przestrzeni nazw
\end{itemize}
}

\frame[containsverbatim]{
\frametitle{Scalamy wspólne prefixy}

Dla kolidujących definicji makr

\begin{verbatim}
macro if1 (cond, expr)
syntax ("if", "(", cond, ")", expr) { ... }

macro if2 (cond, expr1, expr2)
syntax ("if", "(", cond, ")", expr1, "else", expr2) { ... }
\end{verbatim}

budujemy drzewo i podczas parsowania próbujemy zawsze wybrać najdłuższą pasującą
ścieżkę.

\begin{verbatim}
"if" "(" wyrażenie ")" wyrażenie --> koniec
                                 --> "else" wyrażenie koniec
\end{verbatim}
}

\section{Opóźnianie parsowania}

\frame[containsverbatim]{
\frametitle{Chcemy czegoś więcej}
Poprzednie podejście jest dosyć ograniczone, każde rozszerzenie jest zasadniczo
pojedynczą produkcją, którą dokładamy do języka. \\
\vspace*{2mm}
My zaś chcemy elastycznego systemu, dzięku któremu jesteśmy w stanie wbudować 
dowolny podjęzyk w składnię Nemerle.

\begin{verbatim}
def document = xml <person>
  <name>Ala</name>
  <city>Wrocław</city>
</person>;

print (document.InnerXml);

def expr = regex ^+[^-]*-file\.n;
...
\end{verbatim}
}

\frame[containsverbatim]{
\frametitle{Faza preparsowania}
  Pomiędzy fazą leksera i parserem wprowadzamy dodatkowy etap, w którym wstępnie
  przetwarzamy ciąg tokenów. Budujemy pośrednie drzewo, w którym nawiasy, klamry, itp. są
  sparowanie w grupy. 

\begin{verbatim}
 fun f (x : string) {
   def y = System.Int32.Parse (x);
   y + 1
 }
\end{verbatim}

  przekształcany jest w drzewo, w którym każdy węzeł odpowiadający nawiasom
  zawiera listę grup tokenów oddzielonych separatorami. Dla \verb,(), separatorem
  jest precinek \verb.,. a dla \verb,{}, średnik \verb,;,.

\begin{verbatim}
 [ 'fun' , 'f' , ( [ 'x' , ':' , 'string' ] ) , {
    [ 'def' , 'y' , '=' , 'System' , '.' , 'Int32' , '.' , 
      'Parse' , ( [ 'x' ] ) ] ,
    [ 'y' , '+' , '1' ]
  } ]
\end{verbatim}
}

\frame[containsverbatim]{
\frametitle{Odkładamy pracę na później}
  Powstałe drzewo jest przetwarzane w sposób leniwy. Parser nie analizuje
  fragmentów wskazanych przez rozszerzenia składniowe, ale tworzy specjalny węzeł
  składniowy, zawierający w sobie fragment strumienia tokenów.\\
  \vspace*{2mm}
  Makro może zarządać niesparsowanego fragmentu programu w parametrze i we
  własny sposób go przeanalizować.

\begin{verbatim}
  macro create_xml (tok : Token) 
  syntax ("xml", tok) 
  {
    <[ 
       def document = XmlDocument ();
       def frag = document.CreateDocumentFragment ();
       $(ParseXml (<[ frag ]>, tok) : string);
       document
    ]>
  }
\end{verbatim} %$
}

\frame{
\frametitle{Wady?}
  \begin{itemize}
    \item potencjalnie, budowanie dodatkowego drzewa może być wolniejsze, jednak
      w praktyce okazuje się, że stanowi mniej niż $2\%$ czasu kompilacji
    \item języki które wbudowujemy w składnię nadal muszą spełniać założenia
      leksykalne Nemerle, czego normalnie uniknęlibyśmy bezpośrednio modyfikując lekser
    \item tracimy własność terminacji analizy składniowej, którą zachowują
      systemy rozszerzeń opierających się na gramatykach
  \end{itemize}
}

\section{To jest już koniec}
\frame{
\frametitle{Podsumowanie}
  \begin{itemize}
    \item makra umożliwiają sprawną generację kodu
    \item w połączeniu z rozszerzeniami składniowymi dają potężne możliwości
      dostosowywania języka do indywidualnych potrzeb programisty / projektu 
    \item istniejące projekty badawcze (np. $C\omega$ z Microsoft Research)
      dodają zbiór rozszerzeń języka, których duża część jest czysto składniowymi
      zmianami i/lub konstrukcjami niewykraczającymi poza możliwości makr
    \item pakiety makr są w pełni modularnym rozwiązaniem - kompilator ładuje je
      jako opcjonalne biblioteki dodające nową funkcjonalność
  \end{itemize}
}


\end{document}

% vim: language=polish
